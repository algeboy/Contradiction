\documentclass{amsart}

\usepackage{amsmath,amssymb,amsrefs,amsthm}


\newtheorem{thm}{Theorem}
\newtheorem{lemma}[thm]{Lemma}
\newtheorem{prop}[thm]{Proposition}

\newtheorem{defn}[thm]{Definition}

\title{Please Don't Contradict Me}
\author{James B. Wilson}
\date{\today}


\begin{document}
    \maketitle

\section{Introduction}
We all have argued something is false by considering the consequences 
of assuming it is true.   The idea occurs throughout cultures and 
eras and so it seems sound.  Mathematics has numerous examples including 
the following claims.
\begin{itemize}
    \item $\sqrt{2}$ is irrational.
    \item There are infinitely many primes.
    \item The decimal numbers outnumber the integers.
\end{itemize}
In each of these we argue by assuming the opposite, for example 
that $\sqrt{2}=a/b$, and we work towards a difficulty we cannot accept.
Many authors and speakers of today describe such proofs as 
\emph{indirect}.  Some even say they are \emph{proofs by contradiction}.
The very bold will invoke latin titles
\emph{reducto ad absurdum}, a nod to the rich history.

The latin titles not withstanding, western philosophy actually spent most of the
last 2 millennia concerned with a poetic schemes for reasoning described by
Socrates and known as \emph{syllogisms}.  There are 256 syllogisms, of which 24
are called valid.  Even the form of syllogisms is different from what 
today's authors mean by indirect proof, or proof by contradiction.  Those are
much more recent concepts due in large part to the polymath David Hilbert, who
published a widely read treatise on the foundations of math in 1927.
Contemporaries to Hilbert, namely Brouwer, Heyting, and Kolmogorov (BHK) are far
less known, but their work was almost exclusively on foundations 
which permitted them to notice and craft a far more nuanced ontology of arguments.

Mathematics today is such a broad field with connections to so many areas that
it benefits enormously from the informal and expedient treatment given by
Hilbert's system.  Yet, a growing number of fields including Computer Science
and Linguistics, as well as the mathematical disciplines of Universal Algebra,
Lattices,  Topology, \& Category Theory; are today turning towards the lessons
of his contemporaries.  Arguably the most influential difference is the clarity
that the \emph{BHK interpretation} brings to the study of negatives.

\section{What is a negative?}
In symbolic form a negative is a sentence beginning in negation, 
often written $\neg P$.  In our examples language can serve to 
hide this negation so it is a good practice to begin teasing out 
what is hidden.
\begin{itemize}
    \item It is \emph{not true} that $\sqrt{2}$ is rational.
    \item It is false that the set of primes is finite.
    \item There is no bijection $\mathbb{R}\to \mathbb{Z}$
\end{itemize}
To make this even more stark we can use some symbolic logic.
\begin{itemize}
    \item $P\equiv \sqrt{2}=a/b$.  \textbf{Claim} $\neg P$.
    \item $Q\equiv (\exists n\in \mathbb{N})(|\mathsf{Primes}|=n)$. 
    \textbf{Proposition} $\neg Q$
    \item $R\equiv (\exists f:\mathbb{R}\to \mathbb{Z})(g:\mathbb{Z}\to \mathbb{R})(f\circ g=1)\wedge (g\circ f=1)$.
    \textbf{Theorem} $\neg R$.
\end{itemize}

\subsection*{Choices}
Already you may notice a bit of a predicament.  We seem to have options.  In the
case of primes we appealed to a definition of finite, for instance
\begin{center} 
    ``a set is finite
if it has bijection to a set $\{1,\ldots,n\}$ for some natural number $n\in
\mathbb{N}=\{0,1,\ldots\}$''
\end{center}
We might have wondered instead if we could use a direct definition of 
infinite instead, perhaps Cantor's definition:
\begin{center}
    ``a set with a bijection to a proper subset''.
\end{center}
If we did so then ``There are infinitely many primes'' would cease 
to be a negative.  

Of course if we switch definitions then 
it is on us to also demonstrate that those definitions agree.
There in hides our missing negative.
In our setting, if your approach to proving that there are 
infinitely many primes will somewhere use a phrase like 
\begin{center}
    ``Let $p_1,\ldots,p_n$ be the assumed list of all primes...''
\end{center}
then plainly you are using the first of the two possible definitions 
and your proof is in fact proving a negative.  We shall get to ways to 
prove negatives in a moment but now is a good time to acknowledge our 
problems with the concept.

\section{Psychology of negatives}
``You cannot prove a negative!'' This is a premise often used in arguments both
mindless and seemingly profound.  It is picked up at an early age rooting its
philosophy deep into a our early intuition.  
If it is true we would do well to steer clear of arguments 
that prove negatives and attach different rationalizations to our 
arguments, such as ``proof by contradiction, modus tolens'' and such.
But is this phrase even true?

To begin with the idiom itself is a negative.  It is equivalent to stating:
\begin{center}
    $P\equiv$ ``You can prove a negative''\\
    $\neg P\equiv$ ``You cannot prove a negative.''
\end{center}
So the sentence, if true, would need to be a proof of a
negative, which the sentence argues cannot be done.  So we ought to 
ignore this sentence entirely for logical purposes.

However it is worth acknowledging what this phrase has done 
to readers of proofs.
As psychologist Stephen Law observes in \emph{Psychology Today}, Sept. 15, 2011,
this phrase often adequately summarizes the nature of \emph{doubt}, 
not \emph{truth}.  
As scientists of reason we should like to remove doubt 
and so arguments seen as ``proving a negative'' do indeed deserve 
close scrutiny.

% \subsection{Negative arguments use language similar to logical fallacies}
\subsection*{Negatives arguments read like false arguments.}
One reason for so much doubt in proving negatives is that the arguments lie
adjacent to many known logical fallacies (invalid arguments).  

For instance, here is a flawed conclusion 
\begin{quote}
    \textbf{Arugment from ignorance}
    ``I've never seen aliens; so, they do not exist.''
\end{quote}
Here is a 
\begin{quote}
    
\end{quote}
%     ~\hspace{2in} (\emph{Argument from
%     Ignorance})\\[1pt]
%     ``I know James and he would never have made so many spelling errors.''
%     (\emph{Appeal to Authority})\\[1pt]
%     ``55,723 is prime; don't believe me? You find a proper factor.''
%     (\emph{Burden of Proof})    
% \end{quote}
These are not proofs of anything but they do rather well at 
expressing a degree of belief and doubt.  If we lower the doubt 
below a reader's threshold the result seems indistinguishable from proof.
But it is not, in the end, a proof.

Mathematicians do not escape these issues in their proofs,
but they have instead established conventions on the permitted use 
of these sort of fallacies.  For example, how many proofs can be 
found that argue with phrases like these:
\begin{quote} 
    \emph{It is enough to consider the special case ...}\\
    \emph{...the argument follows similarly for other cases.}\\
    \emph{Clearly...}\\
    \emph{...mutatits mutandis.}
\end{quote}


% Understanding those contours helps us feel confident in the 
% need for the details of the eventual correct notion.


% \subsection*{Argument from Ignorance}
% Consider the following statements.
% \begin{itemize}
%     \item I don't see aliens so they do not exist.
%     \item No prime less than 100 divides 55,723 so it is prime.
%     % \item It is enough to consider the special case where...
% \end{itemize}
% Hopefully we all recognize these claims are insufficient to be taken 
% as truths, rather, they each merely identify a source for doubt.
% Logicians refer to this  as \emph{Argument from Ignorance},
% the ignorance being the failure to know all the evidence before 
% drawing a conclusion. 

% Mathematics does not escape these concerns, but they do 
% a good job of disguising it to simplify arguments. 
% For instance, many well-intentioned 
% proofs use phrases such as 
% \begin{quote} 
%     \emph{It is enough to consider the special case ...}\\
%     \emph{...the argument follows similarly for other cases.}\\
%     \emph{...mutatits mutandis.}
% \end{quote}
% We can appreciate the expedience that such phrasing permits but 
% it is nevertheless the case that these are attempts to leave out 
% vital evidence by passing judgement on the relative value or ease 
% of acquiring further evidence.  But in truth such arguments appeal to results 
% that the evidence provided cannot validate.  


% Such fallacies are easily remedied into factual claims.
% \begin{itemize}
%     \item I don't see aliens so \emph{I doubt} they exist.
%     \item No prime less than 100 divides 55,723 so \emph{I believe} it is prime.
% \end{itemize}
% In mathematics the expected correction is usually of the form 
% ``Leaving an exercise $X$ to the reader, it then follows ...''.
% Under the principle of assuming good intent we are inclined 
% in most setting to grant the author such implicit alterations
% to their statement.

% \subsection*{Burden of Proof}
% Another family of errors that supports misgivings about ``proving a negative''
% comes from claims like these.
% \begin{itemize}
%     \item Aliens do exist unless you show me why they don't.
%     \item 55,723 is prime until you show me a factor.
% \end{itemize}
% Philosopher's refer to this as \emph{shifting the burden of proof}
% and declare it to be mistake of reasoning.  Proofs in mathematics 
% often do the same ending a brief proof with stock phrases such as 
% \begin{quote}
% \emph{...which leads to a contradiction.}
% \end{quote}
% Once more our read must decide what is being contradicted and how.  It is 
% in every sense a shift of the burden of truth but again one where 
% good will makes its use acceptable.

\subsection{Linguistic negatives}
Neither \emph{Argument from Ignorance} nor \emph{Burden of Proof} fallacies are
unique to negatives.  The fundamental struggle is that when proving a negative
we begin precisely doubting what we claim.  So the very language of our argument
entertains the vary vocabulary that often leads us to speak forcefully but in
error.





\subsection{Detecting negatives}

\section{Arguing negatives to a machines}
Since human language has such a p

The actual name is \emph{proving a negative}.

Many books introducing scholars to reasoning do not even mention this 
type of proof by name.  Some authors even make the effort to credit 
historical studies given methods of proofs their latin titles
\emph{modus ponens} and \emph{tolens}, and \emph{reducto ad 
absurdum}.  Have we lost track of the methods of proofIf it seems  did we come so far and loose track of the method 
and name for a proof we use so often?


Knowing the difference makes certain that our arguments on based 
on firm reasoning that ends in the desired conclusion.
Without it we may succeed in an argument that lowers our doubt 
below what we tolerate but leave open the door that others may 
still suspect our claims.





\section{Examples.}

\begin{prop}
   $\sqrt{2}$ is \emph{irrational}. 
\end{prop}
First we identify the negativity.  Here \emph{ir-}rational is a 
linguistic trick to mean \emph{not} rational.  So we actually mean the following.

\begin{prop}[Formal Form]
    $\neg(\sqrt{2}\in \mathbb{Q})$
\end{prop}

Second we identify the context, in this case that means to explain 
some possible meaning of rational.

\begin{defn}
    A \emph{rational number} is a solution to an equation 
    $a=bx$ for integers $a,b$ with $b\neq 0$.
    We write $x\in \mathbb{Q}$.
\end{defn}

So we we break up the proof to see what is happening.
\begin{lemma}\label{lem:irrational}
    If there are integers $a,b$ for which $b\sqrt{2}=a$ then 
    $b=0$.
    % $(\exists a,b\in \mathbb{Z})(b\sqrt{2}=a)\Rightarrow (b=0)$.
\end{lemma}
This lemma is the heart of the proof and can be proved directly
but we leave that proof to an appendix so as to stay focussed 
on negativity.

\begin{prop}[Formal Form]
    $(\sqrt{2}\notin \mathbb{Q})\equiv (\sqrt{2}\in \mathbb{Q}\Rightarrow \bot)$
\end{prop}
\begin{proof}
    Assume $\sqrt{2}\in \mathbb{Q}$.  That means that there are 
    integer $a,b$ such that $a=b\sqrt{2}$ and $b\neq 0$.
    Next by Lemma~\ref{lem:irrational} we also know 
    that if $b\sqrt{2}=a$ then $b=0$. 
    Finally $b\neq 0$ means $(b=0)\Rightarrow \bot$, 
    and $b=0$; so, we deduce $\bot$.
    Dismissing the assumed hypothesis we have shown 
    $(\sqrt{2}\in \mathbb{Q})\Rightarrow \bot$.
    That is, $\sqrt{2}\notin \mathbb{Q}$.
\end{proof}

\begin{prop}
    $(|\mathbb{N}|\neq |\mathbb{R}|)\equiv [(|\mathbb{N}|=|\mathbb{R}|)\Rightarrow \bot]$. 
\end{prop}

\begin{lemma}\label{lem:diagonal}
    If $f:\mathbb{N}\to \mathbb{R}$ is a function then there is 
    an $x^f\in \mathbb{R}$ such that the $k$-th decimal place of 
    $x^f$ is
    \begin{align*}
        x^f_k & = \left\{\begin{array}{cc}
            f(k)_k+1 & f(k)>0\\
            9 & f(k)=0
        \end{array}
        \right.
    \end{align*}
\end{lemma}

This proof is again the heart of the proposition but it is not a negative, 
though in this case case it does have an embedded negative in premise 
``$x$ is not in the image of $f$''.  However the full sentence reads 
 and so this is not a negative itself.

\begin{prop}
    $\neg (\exists f:\mathbb{N}\to \mathbb{R})(\forall x\in \mathbb{R})(\exists k\in\mathbb{Z})(f(k)=x)$
\end{prop}
\begin{proof}
    Assume for there is an $f:\mathbb{N}\to \mathbb{R}$ where 
    for every $x\in\mathbb{R}$ has some $k\in \mathbb{Z}$ such that $f(k)=x$.
    Using the $x^f$ from Lemma~\ref{lem:diagonal} then there is 
    some $k\in \mathbb{Z}$ such that $f(k)=x^f$.  Therefore 
    the $k$-th decimal places of both agree: $f(k)_k = x^f_k$.
    Therefore $f(k)_k=f(k)_k+1$ or $0=f(k)_k=9$.
    In the first case $0=1$ and in the second $0=9$.
    Since $0\neq 1$ and $0\neq 9$ in either case we deduce $\bot$.
    Dismissing the assumption, if there is a surjection $f:\mathbb{N}\twoheadrightarrow \mathbb{R}$
    then $\bot$ follows.  So $\not\exists f:\mathbb{N}\twoheadrightarrow \mathbb{R}$.
\end{proof}

\appendix

\section{Proofs for the curious}


\begin{proof}[Proof of Lemma~\ref{lem:irrational}]
    To prove the implication, assume the hypothesis, in this case 
    that for integers $a,b$ $b\sqrt{2}=a$.
    If $b<0$ then replace with $(-a,-b)$.
    So $S=\{b\in \mathbb{N}\mid (\exists a\in \mathbb{Z})(b\sqrt{2}=a)\}$
    is non-empty.  Let $b\in S$ be the least element of $b$, and 
    let $a$ be such that $b\sqrt{2}=a$.
    Square both sides to find $2b^2=a^2$.  So $2|aa$ so by Euclid's 
    Lemma $2|a$. Thus $4|a^2=2b^2$ and so $2|b^2$ which by 
    Euclid's lemma proves $2|b$.  Therefore $(b/2)\sqrt{2}=(a/2)$ 
    so $b/2\in S$.  Since $b$ is minimum $b/2=b$  Therefore 
    $b=0$.  Since $b\neq 0\equiv (b=0\Rightarrow \bot)$ and $b=0$,
    $\bot$ follows.  Therefore $(b\sqrt{2}=a)\Rightarrow \bot$.
    So $\sqrt{2}$ is irrational.
\end{proof}



\end{document}